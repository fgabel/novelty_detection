\documentclass{mimore}

\newcommand{\sAuthor  }{Frank Gabel, Christopher Klugmann}
\newcommand{\sTitle   }{Novelty Detection }
\newcommand{\sSubtitle}{Finding out-of-distribution objects in semantic segmentation maps}
\newcommand{\sSubject }{3D Computer Vision}
\newcommand{\sCourse  }{HCI}
\newcommand{\sNumber  }{3537204, xxxxxxx}
\newcommand{\sDate    }{\today}

\bibliography{Seminar}

%%%%%%%%%%%%%%%%%%%%%%%%%%%%%%%%%%%%%%%%%%%%%%%%%%%%%%%%%%%%%%%%%%%%%%%%
% TikZ & pgfplots
%%%%%%%%%%%%%%%%%%%%%%%%%%%%%%%%%%%%%%%%%%%%%%%%%%%%%%%%%%%%%%%%%%%%%%%%
\usepackage{makecell}

\usepackage{caption}
\usepackage{tikz}
\usepackage{pgfplots}
\pgfplotsset{compat=1.12}
\usepackage{booktabs}
% Permits accessing the smallest and largest x value of a plot
\makeatletter
  \newcommand{\pgfplotsxmin}{\pgfplots@xmin}
  \newcommand{\pgfplotsxmax}{\pgfplots@xmax}
\makeatother

% Permits accessing the smallest and largest y value of a plot
\makeatletter
  \newcommand{\pgfplotsymin}{\pgfplots@ymin}
  \newcommand{\pgfplotsymax}{\pgfplots@ymax}
\makeatother

%%%%%%%%%%%%%%%%%%%%%%%%%%%%%%%%%%%%%%%%%%%%%%%%%%%%%%%%%%%%%%%%%%%%%%%%
% Begin of main document
%%%%%%%%%%%%%%%%%%%%%%%%%%%%%%%%%%%%%%%%%%%%%%%%%%%%%%%%%%%%%%%%%%%%%%%%

\begin{document}

  \maketitle

  \tableofcontents
  \bookmarksetup{startatroot}

  %%%%%%%%%%%%%%%%%%%%%%%%%%%%%%%%%%%%%%%%%%%%%%%%%%%%%%%%%%%%%%%%%%%%%%
  % Add your sources here. You may also directly input LaTeX commands
  % here, but the use of `\include` is strongly encouraged.
  %%%%%%%%%%%%%%%%%%%%%%%%%%%%%%%%%%%%%%%%%%%%%%%%%%%%%%%%%%%%%%%%%%%%%%

  \section{Introduction}

\subsection{Motivation}

  \section{Methods}

\section{Experiments}
\section{Discussion}
\section{Conclusion}

%%%%%%%%%%%%%%%%%%%%%%%%%%%%%%%%%%%%%%%%%%%%%%%%%%%%%%%%%%%%%%%%%%%%%%%%
% Back matter: don't change anything here
%%%%%%%%%%%%%%%%%%%%%%%%%%%%%%%%%%%%%%%%%%%%%%%%%%%%%%%%%%%%%%%%%%%%%%%%

  \bookmarksetup{startatroot}
  \printbibliography

\end{document}
